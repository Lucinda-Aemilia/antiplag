\documentclass[oneside]{book}
\usepackage[bookmarks=true, colorlinks, linkcolor=gray]{hyperref}
\usepackage[top=1in, bottom=1in, left=1.25in, right=1.25in]{geometry}
\usepackage{cite}
\usepackage{graphicx}
\usepackage{float}
\usepackage{listings}
\usepackage{color}

% Packages required by doxygen
\usepackage{fixltx2e}
\usepackage{calc}
\usepackage{doxygen}
\usepackage[export]{adjustbox} % also loads graphicx
\usepackage{graphicx}
\usepackage[utf8]{inputenc}
\usepackage{makeidx}
\usepackage{multicol}
\usepackage{multirow}
\PassOptionsToPackage{warn}{textcomp}
\usepackage{textcomp}
\usepackage[nointegrals]{wasysym}
\usepackage[table]{xcolor}


% \renewcommand\thesection{\arabic{section}}

\definecolor{mygreen}{rgb}{0,0.6,0}
\definecolor{mygray}{rgb}{0.5,0.5,0.5}
\definecolor{mymauve}{rgb}{0.58,0,0.82}

\lstset{ %
  backgroundcolor=\color{white},   % choose the background color; you must add \usepackage{color} or \usepackage{xcolor}
  basicstyle=\footnotesize\ttfamily,        % the size of the fonts that are used for the code
  breakatwhitespace=false,         % sets if automatic breaks should only happen at whitespace
  breaklines=true,                 % sets automatic line breaking
  captionpos=b,                    % sets the caption-position to bottom
  commentstyle=\color{mygreen}\ttfamily,    % comment style
  frame=single,	                   % adds a frame around the code
  keepspaces=true,                 % keeps spaces in text, useful for keeping indentation of code (possibly needs columns=flexible)
  keywordstyle=\color{blue},       % keyword style
  language=C++,                    % the language of the code
  rulecolor=\color{black},         % if not set, the frame-color may be changed on line-breaks within not-black text (e.g. comments (green here))
  showspaces=false,                % show spaces everywhere adding particular underscores; it overrides 'showstringspaces'
  showstringspaces=false,          % underline spaces within strings only
  showtabs=false,                  % show tabs within strings adding particular underscores
  stepnumber=2,                    % the step between two line-numbers. If it's 1, each line will be numbered
  stringstyle=\color{mymauve},     % string literal style
  tabsize=2,	                   % sets default tabsize to 2 spaces
}

\usepackage{caption}
\captionsetup{labelsep=space,justification=centering,font={bf},singlelinecheck=off,skip=4pt,position=top}

% Add search path
\makeatletter
\providecommand*\input@path{}
\newcommand*\addinputpath[1]{%
  \expandafter\def\expandafter\input@path
  \expandafter{\input@path#1}}%
\addinputpath{{generated/}}
\makeatother

\graphicspath{{generated/}}

% �½ڲ���ҳ
\usepackage{xpatch}
\makeatletter
\xpatchcmd{\chapter}
  {\if@openright\cleardoublepage\else\clearpage\fi}{\par\relax}
  {}{}
\makeatother

\makeatletter
\xpatchcmd{\part}
  {\if@openright\cleardoublepage\else\clearpage\fi}{\par\relax}
  {}{}
\makeatother

\title{AntiPlag Software Design Document}
\author{Zhang Huimeng\\ 2015011280 \and Liu Wentong\\ ????????}

\begin{document}

\maketitle
\author
\clearpage

\tableofcontents
\clearpage


\part{Introduction}

    \chapter{Purpose}

    \chapter{Scope}

    \chapter{Overview}

    \chapter{Reference Material}

    \chapter{Definitions and Acronyms}

\clearpage


\part{System Overview}

\clearpage


\part{System Architecture}

    \chapter{Architectural Design} % ������ϵͳ����Ϊ����ģ�飬����ÿ��ģ��֮��Ĺ��ܺ��໥��ϵ

    \chapter{Decomposition Description}

    \chapter{Design Rationale}

\clearpage


\part{Data Design}

    \chapter{Data Description} % data structure, data storage, process and organization

    \chapter{Data Dictionary} % list the objects and its attributes, methods and method parameters

\clearpage


\part{Component Design}
%--- Begin generated contents ---
\chapter{Namespace Index}
\input{namespaces}
\chapter{Hierarchical Index}
\section{Class Hierarchy}
This inheritance list is sorted roughly, but not completely, alphabetically:\begin{DoxyCompactList}
\item \contentsline{section}{Document}{\pageref{class_document}}{}
\item \contentsline{section}{Homework}{\pageref{class_homework}}{}
\item \contentsline{section}{Pattern}{\pageref{class_pattern}}{}
\item \contentsline{section}{PatternTree}{\pageref{class_pattern_tree}}{}
\item \contentsline{section}{Project}{\pageref{class_project}}{}
\item \contentsline{section}{qt\_meta\_stringdata\_Widget\_t}{\pageref{structqt__meta__stringdata___widget__t}}{}
\item QWidget\begin{DoxyCompactList}
\item \contentsline{section}{Widget}{\pageref{class_widget}}{}
\end{DoxyCompactList}
\item \contentsline{section}{Ui\_Widget}{\pageref{class_ui___widget}}{}
\begin{DoxyCompactList}
\item \contentsline{section}{Ui::Widget}{\pageref{class_ui_1_1_widget}}{}
\end{DoxyCompactList}
\end{DoxyCompactList}

\chapter{Class Index}
\section{Class List}
Here are the classes, structs, unions and interfaces with brief descriptions:\begin{DoxyCompactList}
\item\contentsline{section}{\hyperlink{class_document}{Document} }{\pageref{class_document}}{}
\item\contentsline{section}{\hyperlink{class_homework}{Homework} }{\pageref{class_homework}}{}
\item\contentsline{section}{\hyperlink{class_pattern}{Pattern} \\*Storing patterns from a document }{\pageref{class_pattern}}{}
\item\contentsline{section}{\hyperlink{class_pattern_tree}{PatternTree} }{\pageref{class_pattern_tree}}{}
\item\contentsline{section}{\hyperlink{class_project}{Project} }{\pageref{class_project}}{}
\item\contentsline{section}{\hyperlink{structqt__meta__stringdata___widget__t}{qt\_meta\_stringdata\_Widget\_t} }{\pageref{structqt__meta__stringdata___widget__t}}{}
\item\contentsline{section}{\hyperlink{class_ui___widget}{Ui\_Widget} }{\pageref{class_ui___widget}}{}
\item\contentsline{section}{\hyperlink{class_ui_1_1_widget}{Ui::Widget} }{\pageref{class_ui_1_1_widget}}{}
\item\contentsline{section}{\hyperlink{class_widget}{Widget} }{\pageref{class_widget}}{}
\end{DoxyCompactList}

\chapter{Namespace Documentation}
\input{namespace_ui}
\chapter{Class Documentation}
\hypertarget{class_document}{}\section{Document Class Reference}
\label{class_document}\index{Document@{Document}}
\subsection*{Public Member Functions}
\begin{DoxyCompactItemize}
\item 
\hyperlink{class_document_ad6f3eb7808d3d6ad6f40e5b64a3317bc}{Document} (std\+::string address)
\item 
void \hyperlink{class_document_a607bc11ebda64c08fd22c8ca7aa373d3}{Rabin\+Karp} ()
\begin{DoxyCompactList}\small\item\em Perform Rabin-\/\+Karp algorithm for this document. \end{DoxyCompactList}\item 
void \hyperlink{class_document_aa02349519b475996d1206dfbbcaab349}{K\+MP} ()
\begin{DoxyCompactList}\small\item\em Perdorm K\+MP algorithm for this document. \end{DoxyCompactList}\item 
std\+::string \hyperlink{class_document_aba3c51ac1b4cb63dd9235aaa9e5ab6f0}{get\+Address} ()
\end{DoxyCompactItemize}
\subsection*{Protected Member Functions}
\begin{DoxyCompactItemize}
\item 
void \hyperlink{class_document_a019a44be44bd7b548ea32e4c699bed49}{make\+Pattern} ()
\begin{DoxyCompactList}\small\item\em Make the patterns with winnowing algorithm. \end{DoxyCompactList}\item 
void \hyperlink{class_document_a508bc7255ec6ddf96d0d7078ff0d1229}{preprocess} ()
\item 
bool \hyperlink{class_document_ae81119fe62cb6c73bff4c9283c97884f}{is\+Valid} (char c)
\end{DoxyCompactItemize}
\subsection*{Private Member Functions}
\begin{DoxyCompactItemize}
\item 
\hyperlink{class_document}{Document} \& \hyperlink{class_document_a4ccdb9185bd311793ad385564553153c}{operator=} (const \hyperlink{class_document}{Document} \&other)
\end{DoxyCompactItemize}
\subsection*{Private Attributes}
\begin{DoxyCompactItemize}
\item 
std\+::string \hyperlink{class_document_a20ece3a3a1e633e381d28813ad76a088}{m\+\_\+address}
\item 
std\+::string \hyperlink{class_document_a7d965b64a228ab63761ec4440a5549e0}{m\+\_\+content}
\item 
std\+::vector$<$ \hyperlink{class_pattern}{Pattern} $>$ \hyperlink{class_document_a6cdddf9610a671723515bb6c71553870}{m\+\_\+patterns}
\end{DoxyCompactItemize}


\subsection{Constructor \& Destructor Documentation}
\index{Document@{Document}!Document@{Document}}
\index{Document@{Document}!Document@{Document}}
\subsubsection[{\texorpdfstring{Document(std\+::string address)}{Document(std::string address)}}]{\setlength{\rightskip}{0pt plus 5cm}Document\+::\+Document (
\begin{DoxyParamCaption}
\item[{std\+::string}]{address}
\end{DoxyParamCaption}
)}\hypertarget{class_document_ad6f3eb7808d3d6ad6f40e5b64a3317bc}{}\label{class_document_ad6f3eb7808d3d6ad6f40e5b64a3317bc}


Here is the call graph for this function\+:\nopagebreak
\begin{figure}[H]
\begin{center}
\leavevmode
\includegraphics[width=350pt]{class_document_ad6f3eb7808d3d6ad6f40e5b64a3317bc_cgraph}
\end{center}
\end{figure}




\subsection{Member Function Documentation}
\index{Document@{Document}!get\+Address@{get\+Address}}
\index{get\+Address@{get\+Address}!Document@{Document}}
\subsubsection[{\texorpdfstring{get\+Address()}{getAddress()}}]{\setlength{\rightskip}{0pt plus 5cm}std\+::string Document\+::get\+Address (
\begin{DoxyParamCaption}
{}
\end{DoxyParamCaption}
)\hspace{0.3cm}{\ttfamily [inline]}}\hypertarget{class_document_aba3c51ac1b4cb63dd9235aaa9e5ab6f0}{}\label{class_document_aba3c51ac1b4cb63dd9235aaa9e5ab6f0}


Here is the call graph for this function\+:
\nopagebreak
\begin{figure}[H]
\begin{center}
\leavevmode
\includegraphics[width=350pt]{class_document_aba3c51ac1b4cb63dd9235aaa9e5ab6f0_cgraph}
\end{center}
\end{figure}


\index{Document@{Document}!is\+Valid@{is\+Valid}}
\index{is\+Valid@{is\+Valid}!Document@{Document}}
\subsubsection[{\texorpdfstring{is\+Valid(char c)}{isValid(char c)}}]{\setlength{\rightskip}{0pt plus 5cm}bool Document\+::is\+Valid (
\begin{DoxyParamCaption}
\item[{char}]{c}
\end{DoxyParamCaption}
)\hspace{0.3cm}{\ttfamily [protected]}}\hypertarget{class_document_ae81119fe62cb6c73bff4c9283c97884f}{}\label{class_document_ae81119fe62cb6c73bff4c9283c97884f}


Here is the caller graph for this function\+:\nopagebreak
\begin{figure}[H]
\begin{center}
\leavevmode
\includegraphics[width=350pt]{class_document_ae81119fe62cb6c73bff4c9283c97884f_icgraph}
\end{center}
\end{figure}


\index{Document@{Document}!K\+MP@{K\+MP}}
\index{K\+MP@{K\+MP}!Document@{Document}}
\subsubsection[{\texorpdfstring{K\+M\+P()}{KMP()}}]{\setlength{\rightskip}{0pt plus 5cm}void Document\+::\+K\+MP (
\begin{DoxyParamCaption}
{}
\end{DoxyParamCaption}
)}\hypertarget{class_document_aa02349519b475996d1206dfbbcaab349}{}\label{class_document_aa02349519b475996d1206dfbbcaab349}


Perdorm K\+MP algorithm for this document. 



Here is the call graph for this function\+:\nopagebreak
\begin{figure}[H]
\begin{center}
\leavevmode
\includegraphics[width=350pt]{class_document_aa02349519b475996d1206dfbbcaab349_cgraph}
\end{center}
\end{figure}


\index{Document@{Document}!make\+Pattern@{make\+Pattern}}
\index{make\+Pattern@{make\+Pattern}!Document@{Document}}
\subsubsection[{\texorpdfstring{make\+Pattern()}{makePattern()}}]{\setlength{\rightskip}{0pt plus 5cm}void Document\+::make\+Pattern (
\begin{DoxyParamCaption}
{}
\end{DoxyParamCaption}
)\hspace{0.3cm}{\ttfamily [protected]}}\hypertarget{class_document_a019a44be44bd7b548ea32e4c699bed49}{}\label{class_document_a019a44be44bd7b548ea32e4c699bed49}


Make the patterns with winnowing algorithm. 



Here is the caller graph for this function\+:\nopagebreak
\begin{figure}[H]
\begin{center}
\leavevmode
\includegraphics[width=350pt]{class_document_a019a44be44bd7b548ea32e4c699bed49_icgraph}
\end{center}
\end{figure}


\index{Document@{Document}!operator=@{operator=}}
\index{operator=@{operator=}!Document@{Document}}
\subsubsection[{\texorpdfstring{operator=(const Document \&other)}{operator=(const Document &other)}}]{\setlength{\rightskip}{0pt plus 5cm}{\bf Document}\& Document\+::operator= (
\begin{DoxyParamCaption}
\item[{const {\bf Document} \&}]{other}
\end{DoxyParamCaption}
)\hspace{0.3cm}{\ttfamily [private]}}\hypertarget{class_document_a4ccdb9185bd311793ad385564553153c}{}\label{class_document_a4ccdb9185bd311793ad385564553153c}


Here is the caller graph for this function\+:
\nopagebreak
\begin{figure}[H]
\begin{center}
\leavevmode
\includegraphics[width=340pt]{class_document_a4ccdb9185bd311793ad385564553153c_icgraph}
\end{center}
\end{figure}


\index{Document@{Document}!preprocess@{preprocess}}
\index{preprocess@{preprocess}!Document@{Document}}
\subsubsection[{\texorpdfstring{preprocess()}{preprocess()}}]{\setlength{\rightskip}{0pt plus 5cm}void Document\+::preprocess (
\begin{DoxyParamCaption}
{}
\end{DoxyParamCaption}
)\hspace{0.3cm}{\ttfamily [protected]}}\hypertarget{class_document_a508bc7255ec6ddf96d0d7078ff0d1229}{}\label{class_document_a508bc7255ec6ddf96d0d7078ff0d1229}
Remove spaces and taps and etc; Need to replace comments Need to add replacement 

Here is the call graph for this function\+:\nopagebreak
\begin{figure}[H]
\begin{center}
\leavevmode
\includegraphics[width=327pt]{class_document_a508bc7255ec6ddf96d0d7078ff0d1229_cgraph}
\end{center}
\end{figure}




Here is the caller graph for this function\+:\nopagebreak
\begin{figure}[H]
\begin{center}
\leavevmode
\includegraphics[width=347pt]{class_document_a508bc7255ec6ddf96d0d7078ff0d1229_icgraph}
\end{center}
\end{figure}


\index{Document@{Document}!Rabin\+Karp@{Rabin\+Karp}}
\index{Rabin\+Karp@{Rabin\+Karp}!Document@{Document}}
\subsubsection[{\texorpdfstring{Rabin\+Karp()}{RabinKarp()}}]{\setlength{\rightskip}{0pt plus 5cm}void Document\+::\+Rabin\+Karp (
\begin{DoxyParamCaption}
{}
\end{DoxyParamCaption}
)}\hypertarget{class_document_a607bc11ebda64c08fd22c8ca7aa373d3}{}\label{class_document_a607bc11ebda64c08fd22c8ca7aa373d3}


Perform Rabin-\/\+Karp algorithm for this document. 



Here is the call graph for this function\+:\nopagebreak
\begin{figure}[H]
\begin{center}
\leavevmode
\includegraphics[width=350pt]{class_document_a607bc11ebda64c08fd22c8ca7aa373d3_cgraph}
\end{center}
\end{figure}




\subsection{Member Data Documentation}
\index{Document@{Document}!m\+\_\+address@{m\+\_\+address}}
\index{m\+\_\+address@{m\+\_\+address}!Document@{Document}}
\subsubsection[{\texorpdfstring{m\+\_\+address}{m_address}}]{\setlength{\rightskip}{0pt plus 5cm}std\+::string Document\+::m\+\_\+address\hspace{0.3cm}{\ttfamily [private]}}\hypertarget{class_document_a20ece3a3a1e633e381d28813ad76a088}{}\label{class_document_a20ece3a3a1e633e381d28813ad76a088}
\index{Document@{Document}!m\+\_\+content@{m\+\_\+content}}
\index{m\+\_\+content@{m\+\_\+content}!Document@{Document}}
\subsubsection[{\texorpdfstring{m\+\_\+content}{m_content}}]{\setlength{\rightskip}{0pt plus 5cm}std\+::string Document\+::m\+\_\+content\hspace{0.3cm}{\ttfamily [private]}}\hypertarget{class_document_a7d965b64a228ab63761ec4440a5549e0}{}\label{class_document_a7d965b64a228ab63761ec4440a5549e0}
\index{Document@{Document}!m\+\_\+patterns@{m\+\_\+patterns}}
\index{m\+\_\+patterns@{m\+\_\+patterns}!Document@{Document}}
\subsubsection[{\texorpdfstring{m\+\_\+patterns}{m_patterns}}]{\setlength{\rightskip}{0pt plus 5cm}std\+::vector$<${\bf Pattern}$>$ Document\+::m\+\_\+patterns\hspace{0.3cm}{\ttfamily [private]}}\hypertarget{class_document_a6cdddf9610a671723515bb6c71553870}{}\label{class_document_a6cdddf9610a671723515bb6c71553870}

\hypertarget{class_homework}{}\section{Homework Class Reference}
\label{class_homework}\index{Homework@{Homework}}
\subsection*{Public Types}
\begin{DoxyCompactItemize}
\item 
enum \hyperlink{class_homework_a6509dc051b3763d7fea21b4557d3e79e}{Homework\+Type} \{ \hyperlink{class_homework_a6509dc051b3763d7fea21b4557d3e79ea36e19dfd1e533060c7fe573d957186b7}{Single}, 
\hyperlink{class_homework_a6509dc051b3763d7fea21b4557d3e79ea70ece26b542060114ccdffac5e147ee9}{Multiple}
 \}
\end{DoxyCompactItemize}
\subsection*{Public Member Functions}
\begin{DoxyCompactItemize}
\item 
\hyperlink{class_homework_aff55dea2a7a6958a7c05c1e843061f81}{Homework} (std\+::string path, \hyperlink{class_homework_a6509dc051b3763d7fea21b4557d3e79e}{Homework\+Type} type)
\begin{DoxyCompactList}\small\item\em Initialization and build the whole file system. \end{DoxyCompactList}\end{DoxyCompactItemize}
\subsection*{Protected Member Functions}
\begin{DoxyCompactItemize}
\item 
bool \hyperlink{class_homework_aab839a8b3d15dbf849b018e2295aecc2}{find\+Single} (std\+::string file\+Path)
\item 
bool \hyperlink{class_homework_aef97adc1d880c7aaf5870d70b05777c0}{find\+Multiple} (std\+::string file\+Path)
\item 
void \hyperlink{class_homework_ae7c7babeff4d2f82b9ac0b139e92701e}{dfs\+Folder} (std\+::string folder\+Path, std\+::vector$<$ std\+::string $>$ \&address)
\end{DoxyCompactItemize}
\subsection*{Private Attributes}
\begin{DoxyCompactItemize}
\item 
\hyperlink{class_homework_a6509dc051b3763d7fea21b4557d3e79e}{Homework\+Type} \hyperlink{class_homework_a2a684368d7bdfaa95ee102e4fa6372ba}{m\+\_\+type}
\item 
std\+::string \hyperlink{class_homework_acfd58f754bd1712de9903dc986dc6d1e}{m\+\_\+path}
\item 
std\+::vector$<$ \hyperlink{class_project}{Project} $>$ \hyperlink{class_homework_a7d9479cfebf02074e2ac340561f7fb8e}{m\+\_\+projects}
\end{DoxyCompactItemize}


\subsection{Member Enumeration Documentation}
\index{Homework@{Homework}!Homework\+Type@{Homework\+Type}}
\index{Homework\+Type@{Homework\+Type}!Homework@{Homework}}
\subsubsection[{\texorpdfstring{Homework\+Type}{HomeworkType}}]{\setlength{\rightskip}{0pt plus 5cm}enum {\bf Homework\+::\+Homework\+Type}}\hypertarget{class_homework_a6509dc051b3763d7fea21b4557d3e79e}{}\label{class_homework_a6509dc051b3763d7fea21b4557d3e79e}
\begin{Desc}
\item[Enumerator]\par
\begin{description}
\index{Single@{Single}!Homework@{Homework}}\index{Homework@{Homework}!Single@{Single}}\item[{\em 
Single\hypertarget{class_homework_a6509dc051b3763d7fea21b4557d3e79ea36e19dfd1e533060c7fe573d957186b7}{}\label{class_homework_a6509dc051b3763d7fea21b4557d3e79ea36e19dfd1e533060c7fe573d957186b7}
}]\index{Multiple@{Multiple}!Homework@{Homework}}\index{Homework@{Homework}!Multiple@{Multiple}}\item[{\em 
Multiple\hypertarget{class_homework_a6509dc051b3763d7fea21b4557d3e79ea70ece26b542060114ccdffac5e147ee9}{}\label{class_homework_a6509dc051b3763d7fea21b4557d3e79ea70ece26b542060114ccdffac5e147ee9}
}]\end{description}
\end{Desc}


\subsection{Constructor \& Destructor Documentation}
\index{Homework@{Homework}!Homework@{Homework}}
\index{Homework@{Homework}!Homework@{Homework}}
\subsubsection[{\texorpdfstring{Homework(std\+::string path, Homework\+Type type)}{Homework(std::string path, HomeworkType type)}}]{\setlength{\rightskip}{0pt plus 5cm}Homework\+::\+Homework (
\begin{DoxyParamCaption}
\item[{std\+::string}]{path, }
\item[{{\bf Homework\+Type}}]{type}
\end{DoxyParamCaption}
)}\hypertarget{class_homework_aff55dea2a7a6958a7c05c1e843061f81}{}\label{class_homework_aff55dea2a7a6958a7c05c1e843061f81}


Initialization and build the whole file system. 



Here is the call graph for this function\+:\nopagebreak
\begin{figure}[H]
\begin{center}
\leavevmode
\includegraphics[width=350pt]{class_homework_aff55dea2a7a6958a7c05c1e843061f81_cgraph}
\end{center}
\end{figure}




\subsection{Member Function Documentation}
\index{Homework@{Homework}!dfs\+Folder@{dfs\+Folder}}
\index{dfs\+Folder@{dfs\+Folder}!Homework@{Homework}}
\subsubsection[{\texorpdfstring{dfs\+Folder(std\+::string folder\+Path, std\+::vector$<$ std\+::string $>$ \&address)}{dfsFolder(std::string folderPath, std::vector< std::string > &address)}}]{\setlength{\rightskip}{0pt plus 5cm}void Homework\+::dfs\+Folder (
\begin{DoxyParamCaption}
\item[{std\+::string}]{folder\+Path, }
\item[{std\+::vector$<$ std\+::string $>$ \&}]{address}
\end{DoxyParamCaption}
)\hspace{0.3cm}{\ttfamily [protected]}}\hypertarget{class_homework_ae7c7babeff4d2f82b9ac0b139e92701e}{}\label{class_homework_ae7c7babeff4d2f82b9ac0b139e92701e}


Here is the caller graph for this function\+:\nopagebreak
\begin{figure}[H]
\begin{center}
\leavevmode
\includegraphics[width=350pt]{class_homework_ae7c7babeff4d2f82b9ac0b139e92701e_icgraph}
\end{center}
\end{figure}


\index{Homework@{Homework}!find\+Multiple@{find\+Multiple}}
\index{find\+Multiple@{find\+Multiple}!Homework@{Homework}}
\subsubsection[{\texorpdfstring{find\+Multiple(std\+::string file\+Path)}{findMultiple(std::string filePath)}}]{\setlength{\rightskip}{0pt plus 5cm}bool Homework\+::find\+Multiple (
\begin{DoxyParamCaption}
\item[{std\+::string}]{file\+Path}
\end{DoxyParamCaption}
)\hspace{0.3cm}{\ttfamily [protected]}}\hypertarget{class_homework_aef97adc1d880c7aaf5870d70b05777c0}{}\label{class_homework_aef97adc1d880c7aaf5870d70b05777c0}


Here is the call graph for this function\+:\nopagebreak
\begin{figure}[H]
\begin{center}
\leavevmode
\includegraphics[width=341pt]{class_homework_aef97adc1d880c7aaf5870d70b05777c0_cgraph}
\end{center}
\end{figure}




Here is the caller graph for this function\+:\nopagebreak
\begin{figure}[H]
\begin{center}
\leavevmode
\includegraphics[width=347pt]{class_homework_aef97adc1d880c7aaf5870d70b05777c0_icgraph}
\end{center}
\end{figure}


\index{Homework@{Homework}!find\+Single@{find\+Single}}
\index{find\+Single@{find\+Single}!Homework@{Homework}}
\subsubsection[{\texorpdfstring{find\+Single(std\+::string file\+Path)}{findSingle(std::string filePath)}}]{\setlength{\rightskip}{0pt plus 5cm}bool Homework\+::find\+Single (
\begin{DoxyParamCaption}
\item[{std\+::string}]{file\+Path}
\end{DoxyParamCaption}
)\hspace{0.3cm}{\ttfamily [protected]}}\hypertarget{class_homework_aab839a8b3d15dbf849b018e2295aecc2}{}\label{class_homework_aab839a8b3d15dbf849b018e2295aecc2}


Here is the caller graph for this function\+:\nopagebreak
\begin{figure}[H]
\begin{center}
\leavevmode
\includegraphics[width=340pt]{class_homework_aab839a8b3d15dbf849b018e2295aecc2_icgraph}
\end{center}
\end{figure}




\subsection{Member Data Documentation}
\index{Homework@{Homework}!m\+\_\+path@{m\+\_\+path}}
\index{m\+\_\+path@{m\+\_\+path}!Homework@{Homework}}
\subsubsection[{\texorpdfstring{m\+\_\+path}{m_path}}]{\setlength{\rightskip}{0pt plus 5cm}std\+::string Homework\+::m\+\_\+path\hspace{0.3cm}{\ttfamily [private]}}\hypertarget{class_homework_acfd58f754bd1712de9903dc986dc6d1e}{}\label{class_homework_acfd58f754bd1712de9903dc986dc6d1e}
\index{Homework@{Homework}!m\+\_\+projects@{m\+\_\+projects}}
\index{m\+\_\+projects@{m\+\_\+projects}!Homework@{Homework}}
\subsubsection[{\texorpdfstring{m\+\_\+projects}{m_projects}}]{\setlength{\rightskip}{0pt plus 5cm}std\+::vector$<${\bf Project}$>$ Homework\+::m\+\_\+projects\hspace{0.3cm}{\ttfamily [private]}}\hypertarget{class_homework_a7d9479cfebf02074e2ac340561f7fb8e}{}\label{class_homework_a7d9479cfebf02074e2ac340561f7fb8e}
\index{Homework@{Homework}!m\+\_\+type@{m\+\_\+type}}
\index{m\+\_\+type@{m\+\_\+type}!Homework@{Homework}}
\subsubsection[{\texorpdfstring{m\+\_\+type}{m_type}}]{\setlength{\rightskip}{0pt plus 5cm}{\bf Homework\+Type} Homework\+::m\+\_\+type\hspace{0.3cm}{\ttfamily [private]}}\hypertarget{class_homework_a2a684368d7bdfaa95ee102e4fa6372ba}{}\label{class_homework_a2a684368d7bdfaa95ee102e4fa6372ba}

\hypertarget{class_pattern}{}\section{Pattern Class Reference}
\label{class_pattern}\index{Pattern@{Pattern}}


Storing patterns from a document.  




Collaboration diagram for Pattern:
\nopagebreak
\begin{figure}[H]
\begin{center}
\leavevmode
\includegraphics[width=186pt]{class_pattern__coll__graph}
\end{center}
\end{figure}
\subsection*{Public Member Functions}
\begin{DoxyCompactItemize}
\item 
\hyperlink{class_pattern_ada73ecc1fc5e4d94f30a8161feef67cf}{Pattern} (\hyperlink{class_document}{Document} \&parDocument, std::string pattern, int pos)
\begin{DoxyCompactList}\small\item\em The only constructor. \end{DoxyCompactList}\item 
bool \hyperlink{class_pattern_a157a34771b4c550a7cf528f09fa685b5}{operator$<$} (const \hyperlink{class_pattern}{Pattern} \&right)
\begin{DoxyCompactList}\small\item\em operator $<$ for inserting into \hyperlink{class_pattern_tree}{PatternTree} \end{DoxyCompactList}\item 
long long int \hyperlink{class_pattern_ab6c1a23c63162c8e9bc27061e9370ed1}{getHash} () const 
\item 
int \hyperlink{class_pattern_a371d9dad975d97190b969ee6e77178c0}{getLength} () const 
\item 
\hyperlink{class_document}{Document} $\ast$ \hyperlink{class_pattern_aba85629c60347a4f3c81bd7088af4c55}{getParDocument} ()
\item 
std::string \hyperlink{class_pattern_ad4f3ad6d391eea8c6fd95efdf7f7246d}{getPattern} () const 
\item 
void \hyperlink{class_pattern_a44a70e6aff49a25fa068487e3d9f0fa3}{print} () const 
\begin{DoxyCompactList}\small\item\em print basic information about the pattern \end{DoxyCompactList}\end{DoxyCompactItemize}
\subsection*{Protected Member Functions}
\begin{DoxyCompactItemize}
\item 
void \hyperlink{class_pattern_a8c7f0e27f620c00de00d097c84e9d6c3}{calcHash} ()
\end{DoxyCompactItemize}
\subsection*{Private Member Functions}
\begin{DoxyCompactItemize}
\item 
\hyperlink{class_pattern}{Pattern} \& \hyperlink{class_pattern_a69f59394d218d0e476ef9259130e6bef}{operator=} (const \hyperlink{class_pattern}{Pattern} \&other)
\end{DoxyCompactItemize}
\subsection*{Private Attributes}
\begin{DoxyCompactItemize}
\item 
std::string \hyperlink{class_pattern_a492ef4124f2dfebee1babe983cf3f726}{m\_pattern}
\item 
int \hyperlink{class_pattern_aa5b42830eafc550988e9156e4c370f67}{m\_pos}
\item 
long long int \hyperlink{class_pattern_ad2d90cf1c416dd89f29e7da860a02bfb}{m\_hash}
\item 
\hyperlink{class_document}{Document} $\ast$ \hyperlink{class_pattern_a3ce8a1a5ba37412027278315935c06b8}{m\_parDocument}
\end{DoxyCompactItemize}


\subsection{Detailed Description}
Storing patterns from a document. 

\subsection{Constructor \& Destructor Documentation}
\index{Pattern@{Pattern}!Pattern@{Pattern}}
\index{Pattern@{Pattern}!Pattern@{Pattern}}
\subsubsection[{\texorpdfstring{Pattern(Document \&parDocument, std::string pattern, int pos)}{Pattern(Document &parDocument, std::string pattern, int pos)}}]{\setlength{\rightskip}{0pt plus 5cm}Pattern::Pattern (
\begin{DoxyParamCaption}
\item[{{\bf Document} \&}]{parDocument, }
\item[{std::string}]{pattern, }
\item[{int}]{pos}
\end{DoxyParamCaption}
)}\hypertarget{class_pattern_ada73ecc1fc5e4d94f30a8161feef67cf}{}\label{class_pattern_ada73ecc1fc5e4d94f30a8161feef67cf}


The only constructor. 



Here is the call graph for this function:\nopagebreak
\begin{figure}[H]
\begin{center}
\leavevmode
\includegraphics[width=297pt]{class_pattern_ada73ecc1fc5e4d94f30a8161feef67cf_cgraph}
\end{center}
\end{figure}




\subsection{Member Function Documentation}
\index{Pattern@{Pattern}!calcHash@{calcHash}}
\index{calcHash@{calcHash}!Pattern@{Pattern}}
\subsubsection[{\texorpdfstring{calcHash()}{calcHash()}}]{\setlength{\rightskip}{0pt plus 5cm}void Pattern::calcHash (
\begin{DoxyParamCaption}
{}
\end{DoxyParamCaption}
)\hspace{0.3cm}{\ttfamily [protected]}}\hypertarget{class_pattern_a8c7f0e27f620c00de00d097c84e9d6c3}{}\label{class_pattern_a8c7f0e27f620c00de00d097c84e9d6c3}


Here is the caller graph for this function:\nopagebreak
\begin{figure}[H]
\begin{center}
\leavevmode
\includegraphics[width=310pt]{class_pattern_a8c7f0e27f620c00de00d097c84e9d6c3_icgraph}
\end{center}
\end{figure}


\index{Pattern@{Pattern}!getHash@{getHash}}
\index{getHash@{getHash}!Pattern@{Pattern}}
\subsubsection[{\texorpdfstring{getHash() const }{getHash() const }}]{\setlength{\rightskip}{0pt plus 5cm}long long int Pattern::getHash (
\begin{DoxyParamCaption}
{}
\end{DoxyParamCaption}
) const\hspace{0.3cm}{\ttfamily [inline]}}\hypertarget{class_pattern_ab6c1a23c63162c8e9bc27061e9370ed1}{}\label{class_pattern_ab6c1a23c63162c8e9bc27061e9370ed1}
\begin{DoxyReturn}{Returns}
The hash value. 
\end{DoxyReturn}


Here is the caller graph for this function:\nopagebreak
\begin{figure}[H]
\begin{center}
\leavevmode
\includegraphics[width=350pt]{class_pattern_ab6c1a23c63162c8e9bc27061e9370ed1_icgraph}
\end{center}
\end{figure}


\index{Pattern@{Pattern}!getLength@{getLength}}
\index{getLength@{getLength}!Pattern@{Pattern}}
\subsubsection[{\texorpdfstring{getLength() const }{getLength() const }}]{\setlength{\rightskip}{0pt plus 5cm}int Pattern::getLength (
\begin{DoxyParamCaption}
{}
\end{DoxyParamCaption}
) const\hspace{0.3cm}{\ttfamily [inline]}}\hypertarget{class_pattern_a371d9dad975d97190b969ee6e77178c0}{}\label{class_pattern_a371d9dad975d97190b969ee6e77178c0}
\begin{DoxyReturn}{Returns}
the length of the pattern. 
\end{DoxyReturn}
\index{Pattern@{Pattern}!getParDocument@{getParDocument}}
\index{getParDocument@{getParDocument}!Pattern@{Pattern}}
\subsubsection[{\texorpdfstring{getParDocument()}{getParDocument()}}]{\setlength{\rightskip}{0pt plus 5cm}{\bf Document}$\ast$ Pattern::getParDocument (
\begin{DoxyParamCaption}
{}
\end{DoxyParamCaption}
)\hspace{0.3cm}{\ttfamily [inline]}}\hypertarget{class_pattern_aba85629c60347a4f3c81bd7088af4c55}{}\label{class_pattern_aba85629c60347a4f3c81bd7088af4c55}
\begin{DoxyReturn}{Returns}
the address of the parent document. Note: need to use const here 
\end{DoxyReturn}
\index{Pattern@{Pattern}!getPattern@{getPattern}}
\index{getPattern@{getPattern}!Pattern@{Pattern}}
\subsubsection[{\texorpdfstring{getPattern() const }{getPattern() const }}]{\setlength{\rightskip}{0pt plus 5cm}std::string Pattern::getPattern (
\begin{DoxyParamCaption}
{}
\end{DoxyParamCaption}
) const\hspace{0.3cm}{\ttfamily [inline]}}\hypertarget{class_pattern_ad4f3ad6d391eea8c6fd95efdf7f7246d}{}\label{class_pattern_ad4f3ad6d391eea8c6fd95efdf7f7246d}


Here is the call graph for this function:
\nopagebreak
\begin{figure}[H]
\begin{center}
\leavevmode
\includegraphics[width=310pt]{class_pattern_ad4f3ad6d391eea8c6fd95efdf7f7246d_cgraph}
\end{center}
\end{figure}


\index{Pattern@{Pattern}!operator$<$@{operator$<$}}
\index{operator$<$@{operator$<$}!Pattern@{Pattern}}
\subsubsection[{\texorpdfstring{operator$<$(const Pattern \&right)}{operator<(const Pattern &right)}}]{\setlength{\rightskip}{0pt plus 5cm}bool Pattern::operator$<$ (
\begin{DoxyParamCaption}
\item[{const {\bf Pattern} \&}]{right}
\end{DoxyParamCaption}
)\hspace{0.3cm}{\ttfamily [inline]}}\hypertarget{class_pattern_a157a34771b4c550a7cf528f09fa685b5}{}\label{class_pattern_a157a34771b4c550a7cf528f09fa685b5}


operator $<$ for inserting into \hyperlink{class_pattern_tree}{PatternTree} 

\index{Pattern@{Pattern}!operator=@{operator=}}
\index{operator=@{operator=}!Pattern@{Pattern}}
\subsubsection[{\texorpdfstring{operator=(const Pattern \&other)}{operator=(const Pattern &other)}}]{\setlength{\rightskip}{0pt plus 5cm}{\bf Pattern}\& Pattern::operator= (
\begin{DoxyParamCaption}
\item[{const {\bf Pattern} \&}]{other}
\end{DoxyParamCaption}
)\hspace{0.3cm}{\ttfamily [private]}}\hypertarget{class_pattern_a69f59394d218d0e476ef9259130e6bef}{}\label{class_pattern_a69f59394d218d0e476ef9259130e6bef}


Here is the caller graph for this function:
\nopagebreak
\begin{figure}[H]
\begin{center}
\leavevmode
\includegraphics[width=310pt]{class_pattern_a69f59394d218d0e476ef9259130e6bef_icgraph}
\end{center}
\end{figure}


\index{Pattern@{Pattern}!print@{print}}
\index{print@{print}!Pattern@{Pattern}}
\subsubsection[{\texorpdfstring{print() const }{print() const }}]{\setlength{\rightskip}{0pt plus 5cm}void Pattern::print (
\begin{DoxyParamCaption}
{}
\end{DoxyParamCaption}
) const}\hypertarget{class_pattern_a44a70e6aff49a25fa068487e3d9f0fa3}{}\label{class_pattern_a44a70e6aff49a25fa068487e3d9f0fa3}


print basic information about the pattern 



Here is the caller graph for this function:\nopagebreak
\begin{figure}[H]
\begin{center}
\leavevmode
\includegraphics[width=288pt]{class_pattern_a44a70e6aff49a25fa068487e3d9f0fa3_icgraph}
\end{center}
\end{figure}




\subsection{Member Data Documentation}
\index{Pattern@{Pattern}!m\_hash@{m\_hash}}
\index{m\_hash@{m\_hash}!Pattern@{Pattern}}
\subsubsection[{\texorpdfstring{m\_hash}{m_hash}}]{\setlength{\rightskip}{0pt plus 5cm}long long int Pattern::m\_hash\hspace{0.3cm}{\ttfamily [private]}}\hypertarget{class_pattern_ad2d90cf1c416dd89f29e7da860a02bfb}{}\label{class_pattern_ad2d90cf1c416dd89f29e7da860a02bfb}
\index{Pattern@{Pattern}!m\_parDocument@{m\_parDocument}}
\index{m\_parDocument@{m\_parDocument}!Pattern@{Pattern}}
\subsubsection[{\texorpdfstring{m\_parDocument}{m_parDocument}}]{\setlength{\rightskip}{0pt plus 5cm}{\bf Document}$\ast$ Pattern::m\_parDocument\hspace{0.3cm}{\ttfamily [private]}}\hypertarget{class_pattern_a3ce8a1a5ba37412027278315935c06b8}{}\label{class_pattern_a3ce8a1a5ba37412027278315935c06b8}
\index{Pattern@{Pattern}!m\_pattern@{m\_pattern}}
\index{m\_pattern@{m\_pattern}!Pattern@{Pattern}}
\subsubsection[{\texorpdfstring{m\_pattern}{m_pattern}}]{\setlength{\rightskip}{0pt plus 5cm}std::string Pattern::m\_pattern\hspace{0.3cm}{\ttfamily [private]}}\hypertarget{class_pattern_a492ef4124f2dfebee1babe983cf3f726}{}\label{class_pattern_a492ef4124f2dfebee1babe983cf3f726}
\index{Pattern@{Pattern}!m\_pos@{m\_pos}}
\index{m\_pos@{m\_pos}!Pattern@{Pattern}}
\subsubsection[{\texorpdfstring{m\_pos}{m_pos}}]{\setlength{\rightskip}{0pt plus 5cm}int Pattern::m\_pos\hspace{0.3cm}{\ttfamily [private]}}\hypertarget{class_pattern_aa5b42830eafc550988e9156e4c370f67}{}\label{class_pattern_aa5b42830eafc550988e9156e4c370f67}

\hypertarget{class_pattern_tree}{}\section{PatternTree Class Reference}
\label{class_pattern_tree}\index{PatternTree@{PatternTree}}


Collaboration diagram for PatternTree:
\nopagebreak
\begin{figure}[H]
\begin{center}
\leavevmode
\includegraphics[width=219pt]{class_pattern_tree__coll__graph}
\end{center}
\end{figure}
\subsection*{Public Member Functions}
\begin{DoxyCompactItemize}
\item 
\hyperlink{class_pattern_tree_a8c393d9dc7966220803f670062d81cb5}{PatternTree} ()
\item 
void \hyperlink{class_pattern_tree_aeedfa28c34ce69dc494675e9662bd60b}{destroy} ()
\item 
void \hyperlink{class_pattern_tree_af698c6d454803b5debff21fe19eecab5}{insert} (const \hyperlink{class_pattern}{Pattern} \&pattern)
\begin{DoxyCompactList}\small\item\em insert a pattern into the tree \end{DoxyCompactList}\item 
std::vector$<$ \hyperlink{class_pattern}{Pattern} $>$ \hyperlink{class_pattern_tree_a86ae88dfb3fd379e2b914590b5a2f896}{find} (const long long int hash)
\item 
std::vector$<$ \hyperlink{class_pattern}{Pattern} $>$ \hyperlink{class_pattern_tree_a01a7afad6ad98b8d5958096c54c45b1f}{getAll} ()
\item 
void \hyperlink{class_pattern_tree_a4db45bd0a698e9999c743f5ac05c33a6}{print} ()
\begin{DoxyCompactList}\small\item\em print some basic information about the tree \end{DoxyCompactList}\end{DoxyCompactItemize}
\subsection*{Static Public Member Functions}
\begin{DoxyCompactItemize}
\item 
static \hyperlink{class_pattern_tree}{PatternTree} $\ast$ \hyperlink{class_pattern_tree_ae3cb1d962789f1bfe37994eec207d493}{instance} ()
\end{DoxyCompactItemize}
\subsection*{Private Member Functions}
\begin{DoxyCompactItemize}
\item 
\hyperlink{class_pattern_tree_af84e0bf107c0fda11f1c971bfe3d9c4f}{PatternTree} (const \hyperlink{class_pattern_tree}{PatternTree} \&other)
\item 
\hyperlink{class_pattern_tree}{PatternTree} \& \hyperlink{class_pattern_tree_ad1ece378133ff4abb282cea9204b0ccc}{operator=} (const \hyperlink{class_pattern_tree}{PatternTree} \&right)
\end{DoxyCompactItemize}
\subsection*{Private Attributes}
\begin{DoxyCompactItemize}
\item 
patternMmap \hyperlink{class_pattern_tree_a8aa612fc369e9106116ad8e3e5b02021}{m\_tree}
\end{DoxyCompactItemize}
\subsection*{Static Private Attributes}
\begin{DoxyCompactItemize}
\item 
static \hyperlink{class_pattern_tree}{PatternTree} $\ast$ \hyperlink{class_pattern_tree_a84a86ffd132359390369421d1fb4c594}{m\_instance} = NULL
\end{DoxyCompactItemize}


\subsection{Detailed Description}
A tree with all patterns stored in it Singleton 

\subsection{Constructor \& Destructor Documentation}
\index{PatternTree@{PatternTree}!PatternTree@{PatternTree}}
\index{PatternTree@{PatternTree}!PatternTree@{PatternTree}}
\subsubsection[{\texorpdfstring{PatternTree()}{PatternTree()}}]{\setlength{\rightskip}{0pt plus 5cm}PatternTree::PatternTree (
\begin{DoxyParamCaption}
{}
\end{DoxyParamCaption}
)}\hypertarget{class_pattern_tree_a8c393d9dc7966220803f670062d81cb5}{}\label{class_pattern_tree_a8c393d9dc7966220803f670062d81cb5}


Here is the caller graph for this function:
\nopagebreak
\begin{figure}[H]
\begin{center}
\leavevmode
\includegraphics[width=350pt]{class_pattern_tree_a8c393d9dc7966220803f670062d81cb5_icgraph}
\end{center}
\end{figure}


\index{PatternTree@{PatternTree}!PatternTree@{PatternTree}}
\index{PatternTree@{PatternTree}!PatternTree@{PatternTree}}
\subsubsection[{\texorpdfstring{PatternTree(const PatternTree \&other)}{PatternTree(const PatternTree &other)}}]{\setlength{\rightskip}{0pt plus 5cm}PatternTree::PatternTree (
\begin{DoxyParamCaption}
\item[{const {\bf PatternTree} \&}]{other}
\end{DoxyParamCaption}
)\hspace{0.3cm}{\ttfamily [private]}}\hypertarget{class_pattern_tree_af84e0bf107c0fda11f1c971bfe3d9c4f}{}\label{class_pattern_tree_af84e0bf107c0fda11f1c971bfe3d9c4f}


\subsection{Member Function Documentation}
\index{PatternTree@{PatternTree}!destroy@{destroy}}
\index{destroy@{destroy}!PatternTree@{PatternTree}}
\subsubsection[{\texorpdfstring{destroy()}{destroy()}}]{\setlength{\rightskip}{0pt plus 5cm}void PatternTree::destroy (
\begin{DoxyParamCaption}
{}
\end{DoxyParamCaption}
)}\hypertarget{class_pattern_tree_aeedfa28c34ce69dc494675e9662bd60b}{}\label{class_pattern_tree_aeedfa28c34ce69dc494675e9662bd60b}
\index{PatternTree@{PatternTree}!find@{find}}
\index{find@{find}!PatternTree@{PatternTree}}
\subsubsection[{\texorpdfstring{find(const long long int hash)}{find(const long long int hash)}}]{\setlength{\rightskip}{0pt plus 5cm}std::vector$<$ {\bf Pattern} $>$ PatternTree::find (
\begin{DoxyParamCaption}
\item[{const long long int}]{hash}
\end{DoxyParamCaption}
)}\hypertarget{class_pattern_tree_a86ae88dfb3fd379e2b914590b5a2f896}{}\label{class_pattern_tree_a86ae88dfb3fd379e2b914590b5a2f896}
find a set of patterns with the same hash value in the tree Note: I cannot make it const... \index{PatternTree@{PatternTree}!getAll@{getAll}}
\index{getAll@{getAll}!PatternTree@{PatternTree}}
\subsubsection[{\texorpdfstring{getAll()}{getAll()}}]{\setlength{\rightskip}{0pt plus 5cm}std::vector$<$ {\bf Pattern} $>$ PatternTree::getAll (
\begin{DoxyParamCaption}
{}
\end{DoxyParamCaption}
)}\hypertarget{class_pattern_tree_a01a7afad6ad98b8d5958096c54c45b1f}{}\label{class_pattern_tree_a01a7afad6ad98b8d5958096c54c45b1f}
\index{PatternTree@{PatternTree}!insert@{insert}}
\index{insert@{insert}!PatternTree@{PatternTree}}
\subsubsection[{\texorpdfstring{insert(const Pattern \&pattern)}{insert(const Pattern &pattern)}}]{\setlength{\rightskip}{0pt plus 5cm}void PatternTree::insert (
\begin{DoxyParamCaption}
\item[{const {\bf Pattern} \&}]{pattern}
\end{DoxyParamCaption}
)}\hypertarget{class_pattern_tree_af698c6d454803b5debff21fe19eecab5}{}\label{class_pattern_tree_af698c6d454803b5debff21fe19eecab5}


insert a pattern into the tree 



Here is the call graph for this function:\nopagebreak
\begin{figure}[H]
\begin{center}
\leavevmode
\includegraphics[width=303pt]{class_pattern_tree_af698c6d454803b5debff21fe19eecab5_cgraph}
\end{center}
\end{figure}




Here is the caller graph for this function:\nopagebreak
\begin{figure}[H]
\begin{center}
\leavevmode
\includegraphics[width=324pt]{class_pattern_tree_af698c6d454803b5debff21fe19eecab5_icgraph}
\end{center}
\end{figure}


\index{PatternTree@{PatternTree}!instance@{instance}}
\index{instance@{instance}!PatternTree@{PatternTree}}
\subsubsection[{\texorpdfstring{instance()}{instance()}}]{\setlength{\rightskip}{0pt plus 5cm}{\bf PatternTree} $\ast$ PatternTree::instance (
\begin{DoxyParamCaption}
{}
\end{DoxyParamCaption}
)\hspace{0.3cm}{\ttfamily [static]}}\hypertarget{class_pattern_tree_ae3cb1d962789f1bfe37994eec207d493}{}\label{class_pattern_tree_ae3cb1d962789f1bfe37994eec207d493}


Here is the call graph for this function:\nopagebreak
\begin{figure}[H]
\begin{center}
\leavevmode
\includegraphics[width=349pt]{class_pattern_tree_ae3cb1d962789f1bfe37994eec207d493_cgraph}
\end{center}
\end{figure}




Here is the caller graph for this function:
\nopagebreak
\begin{figure}[H]
\begin{center}
\leavevmode
\includegraphics[width=338pt]{class_pattern_tree_ae3cb1d962789f1bfe37994eec207d493_icgraph}
\end{center}
\end{figure}


\index{PatternTree@{PatternTree}!operator=@{operator=}}
\index{operator=@{operator=}!PatternTree@{PatternTree}}
\subsubsection[{\texorpdfstring{operator=(const PatternTree \&right)}{operator=(const PatternTree &right)}}]{\setlength{\rightskip}{0pt plus 5cm}{\bf PatternTree}\& PatternTree::operator= (
\begin{DoxyParamCaption}
\item[{const {\bf PatternTree} \&}]{right}
\end{DoxyParamCaption}
)\hspace{0.3cm}{\ttfamily [private]}}\hypertarget{class_pattern_tree_ad1ece378133ff4abb282cea9204b0ccc}{}\label{class_pattern_tree_ad1ece378133ff4abb282cea9204b0ccc}
\index{PatternTree@{PatternTree}!print@{print}}
\index{print@{print}!PatternTree@{PatternTree}}
\subsubsection[{\texorpdfstring{print()}{print()}}]{\setlength{\rightskip}{0pt plus 5cm}void PatternTree::print (
\begin{DoxyParamCaption}
{}
\end{DoxyParamCaption}
)}\hypertarget{class_pattern_tree_a4db45bd0a698e9999c743f5ac05c33a6}{}\label{class_pattern_tree_a4db45bd0a698e9999c743f5ac05c33a6}


print some basic information about the tree 



Here is the caller graph for this function:\nopagebreak
\begin{figure}[H]
\begin{center}
\leavevmode
\includegraphics[width=319pt]{class_pattern_tree_a4db45bd0a698e9999c743f5ac05c33a6_icgraph}
\end{center}
\end{figure}




\subsection{Member Data Documentation}
\index{PatternTree@{PatternTree}!m\_instance@{m\_instance}}
\index{m\_instance@{m\_instance}!PatternTree@{PatternTree}}
\subsubsection[{\texorpdfstring{m\_instance}{m_instance}}]{\setlength{\rightskip}{0pt plus 5cm}{\bf PatternTree} $\ast$ PatternTree::m\_instance = NULL\hspace{0.3cm}{\ttfamily [static]}, {\ttfamily [private]}}\hypertarget{class_pattern_tree_a84a86ffd132359390369421d1fb4c594}{}\label{class_pattern_tree_a84a86ffd132359390369421d1fb4c594}
\index{PatternTree@{PatternTree}!m\_tree@{m\_tree}}
\index{m\_tree@{m\_tree}!PatternTree@{PatternTree}}
\subsubsection[{\texorpdfstring{m\_tree}{m_tree}}]{\setlength{\rightskip}{0pt plus 5cm}patternMmap PatternTree::m\_tree\hspace{0.3cm}{\ttfamily [private]}}\hypertarget{class_pattern_tree_a8aa612fc369e9106116ad8e3e5b02021}{}\label{class_pattern_tree_a8aa612fc369e9106116ad8e3e5b02021}

\hypertarget{class_project}{}\section{Project Class Reference}
\label{class_project}\index{Project@{Project}}
\subsection*{Public Member Functions}
\begin{DoxyCompactItemize}
\item 
\hyperlink{class_project_af9fa9ff9db932f53ccc626940e046bf0}{Project} (std::string path, const std::vector$<$ std::string $>$ \&address)
\begin{DoxyCompactList}\small\item\em Note: need to think about reference here. \end{DoxyCompactList}\end{DoxyCompactItemize}
\subsection*{Private Attributes}
\begin{DoxyCompactItemize}
\item 
std::string \hyperlink{class_project_a6c6b9942014ec60bdc35bfd29d7b21fa}{m\_path}
\item 
std::vector$<$ \hyperlink{class_document}{Document} $>$ \hyperlink{class_project_a9f27e95fa3e22adbdfd6b2789fb9dcf2}{m\_documents}
\end{DoxyCompactItemize}


\subsection{Constructor \& Destructor Documentation}
\index{Project@{Project}!Project@{Project}}
\index{Project@{Project}!Project@{Project}}
\subsubsection[{\texorpdfstring{Project(std::string path, const std::vector$<$ std::string $>$ \&address)}{Project(std::string path, const std::vector< std::string > &address)}}]{\setlength{\rightskip}{0pt plus 5cm}Project::Project (
\begin{DoxyParamCaption}
\item[{std::string}]{path, }
\item[{const std::vector$<$ std::string $>$ \&}]{address}
\end{DoxyParamCaption}
)}\hypertarget{class_project_af9fa9ff9db932f53ccc626940e046bf0}{}\label{class_project_af9fa9ff9db932f53ccc626940e046bf0}


Note: need to think about reference here. 



\subsection{Member Data Documentation}
\index{Project@{Project}!m\_documents@{m\_documents}}
\index{m\_documents@{m\_documents}!Project@{Project}}
\subsubsection[{\texorpdfstring{m\_documents}{m_documents}}]{\setlength{\rightskip}{0pt plus 5cm}std::vector$<${\bf Document}$>$ Project::m\_documents\hspace{0.3cm}{\ttfamily [private]}}\hypertarget{class_project_a9f27e95fa3e22adbdfd6b2789fb9dcf2}{}\label{class_project_a9f27e95fa3e22adbdfd6b2789fb9dcf2}
\index{Project@{Project}!m\_path@{m\_path}}
\index{m\_path@{m\_path}!Project@{Project}}
\subsubsection[{\texorpdfstring{m\_path}{m_path}}]{\setlength{\rightskip}{0pt plus 5cm}std::string Project::m\_path\hspace{0.3cm}{\ttfamily [private]}}\hypertarget{class_project_a6c6b9942014ec60bdc35bfd29d7b21fa}{}\label{class_project_a6c6b9942014ec60bdc35bfd29d7b21fa}

\hypertarget{structqt__meta__stringdata___widget__t}{}\section{qt\_meta\_stringdata\_Widget\_t Struct Reference}
\label{structqt__meta__stringdata___widget__t}\index{qt\_meta\_stringdata\_Widget\_t@{qt\_meta\_stringdata\_Widget\_t}}
\subsection*{Public Attributes}
\begin{DoxyCompactItemize}
\item 
QByteArrayData \hyperlink{structqt__meta__stringdata___widget__t_ae21052b7ec9e52d41a6fad5351c87ffc}{data} \mbox{[}3\mbox{]}
\item 
char \hyperlink{structqt__meta__stringdata___widget__t_a6c44677d08b68447369955e1bddad625}{stringdata0} \mbox{[}14\mbox{]}
\end{DoxyCompactItemize}


\subsection{Member Data Documentation}
\index{qt\_meta\_stringdata\_Widget\_t@{qt\_meta\_stringdata\_Widget\_t}!data@{data}}
\index{data@{data}!qt\_meta\_stringdata\_Widget\_t@{qt\_meta\_stringdata\_Widget\_t}}
\subsubsection[{\texorpdfstring{data}{data}}]{\setlength{\rightskip}{0pt plus 5cm}QByteArrayData qt\_meta\_stringdata\_Widget\_t::data\mbox{[}3\mbox{]}}\hypertarget{structqt__meta__stringdata___widget__t_ae21052b7ec9e52d41a6fad5351c87ffc}{}\label{structqt__meta__stringdata___widget__t_ae21052b7ec9e52d41a6fad5351c87ffc}
\index{qt\_meta\_stringdata\_Widget\_t@{qt\_meta\_stringdata\_Widget\_t}!stringdata0@{stringdata0}}
\index{stringdata0@{stringdata0}!qt\_meta\_stringdata\_Widget\_t@{qt\_meta\_stringdata\_Widget\_t}}
\subsubsection[{\texorpdfstring{stringdata0}{stringdata0}}]{\setlength{\rightskip}{0pt plus 5cm}char qt\_meta\_stringdata\_Widget\_t::stringdata0\mbox{[}14\mbox{]}}\hypertarget{structqt__meta__stringdata___widget__t_a6c44677d08b68447369955e1bddad625}{}\label{structqt__meta__stringdata___widget__t_a6c44677d08b68447369955e1bddad625}

\hypertarget{class_ui___widget}{}\section{Ui\+\_\+\+Widget Class Reference}
\label{class_ui___widget}\index{Ui\+\_\+\+Widget@{Ui\+\_\+\+Widget}}


Inheritance diagram for Ui\+\_\+\+Widget\+:\nopagebreak
\begin{figure}[H]
\begin{center}
\leavevmode
\includegraphics[width=142pt]{class_ui___widget__inherit__graph}
\end{center}
\end{figure}
\subsection*{Public Member Functions}
\begin{DoxyCompactItemize}
\item 
void \hyperlink{class_ui___widget_a9039ed8704971418cbe19ef8c9eea266}{setup\+Ui} (Q\+Widget $\ast$\hyperlink{class_widget}{Widget})
\item 
void \hyperlink{class_ui___widget_ae1cb85db8d3658df8dcd104361edcecb}{retranslate\+Ui} (Q\+Widget $\ast$\hyperlink{class_widget}{Widget})
\end{DoxyCompactItemize}
\subsection*{Public Attributes}
\begin{DoxyCompactItemize}
\item 
Q\+Label $\ast$ \hyperlink{class_ui___widget_a3126b93450dcc18cede73b9d1ee7c6b0}{label}
\item 
Q\+Label $\ast$ \hyperlink{class_ui___widget_a6f06b143349464b5b19ac0ffe2fc084d}{label\+\_\+2}
\item 
Q\+Label $\ast$ \hyperlink{class_ui___widget_adfcab5569ac08da197e14dba01390755}{label\+\_\+3}
\item 
Q\+Push\+Button $\ast$ \hyperlink{class_ui___widget_a7dcf5da8902069415662905e93b0d5cb}{push\+Button}
\item 
Q\+Label $\ast$ \hyperlink{class_ui___widget_a7d22bf9c5cf51754b1c145db5ca0da79}{label\+\_\+4}
\end{DoxyCompactItemize}


\subsection{Member Function Documentation}
\index{Ui\+\_\+\+Widget@{Ui\+\_\+\+Widget}!retranslate\+Ui@{retranslate\+Ui}}
\index{retranslate\+Ui@{retranslate\+Ui}!Ui\+\_\+\+Widget@{Ui\+\_\+\+Widget}}
\subsubsection[{\texorpdfstring{retranslate\+Ui(\+Q\+Widget $\ast$\+Widget)}{retranslateUi(QWidget *Widget)}}]{\setlength{\rightskip}{0pt plus 5cm}void Ui\+\_\+\+Widget\+::retranslate\+Ui (
\begin{DoxyParamCaption}
\item[{Q\+Widget $\ast$}]{Widget}
\end{DoxyParamCaption}
)\hspace{0.3cm}{\ttfamily [inline]}}\hypertarget{class_ui___widget_ae1cb85db8d3658df8dcd104361edcecb}{}\label{class_ui___widget_ae1cb85db8d3658df8dcd104361edcecb}


Here is the caller graph for this function\+:\nopagebreak
\begin{figure}[H]
\begin{center}
\leavevmode
\includegraphics[width=350pt]{class_ui___widget_ae1cb85db8d3658df8dcd104361edcecb_icgraph}
\end{center}
\end{figure}


\index{Ui\+\_\+\+Widget@{Ui\+\_\+\+Widget}!setup\+Ui@{setup\+Ui}}
\index{setup\+Ui@{setup\+Ui}!Ui\+\_\+\+Widget@{Ui\+\_\+\+Widget}}
\subsubsection[{\texorpdfstring{setup\+Ui(\+Q\+Widget $\ast$\+Widget)}{setupUi(QWidget *Widget)}}]{\setlength{\rightskip}{0pt plus 5cm}void Ui\+\_\+\+Widget\+::setup\+Ui (
\begin{DoxyParamCaption}
\item[{Q\+Widget $\ast$}]{Widget}
\end{DoxyParamCaption}
)\hspace{0.3cm}{\ttfamily [inline]}}\hypertarget{class_ui___widget_a9039ed8704971418cbe19ef8c9eea266}{}\label{class_ui___widget_a9039ed8704971418cbe19ef8c9eea266}


Here is the call graph for this function\+:\nopagebreak
\begin{figure}[H]
\begin{center}
\leavevmode
\includegraphics[width=339pt]{class_ui___widget_a9039ed8704971418cbe19ef8c9eea266_cgraph}
\end{center}
\end{figure}




Here is the caller graph for this function\+:\nopagebreak
\begin{figure}[H]
\begin{center}
\leavevmode
\includegraphics[width=301pt]{class_ui___widget_a9039ed8704971418cbe19ef8c9eea266_icgraph}
\end{center}
\end{figure}




\subsection{Member Data Documentation}
\index{Ui\+\_\+\+Widget@{Ui\+\_\+\+Widget}!label@{label}}
\index{label@{label}!Ui\+\_\+\+Widget@{Ui\+\_\+\+Widget}}
\subsubsection[{\texorpdfstring{label}{label}}]{\setlength{\rightskip}{0pt plus 5cm}Q\+Label$\ast$ Ui\+\_\+\+Widget\+::label}\hypertarget{class_ui___widget_a3126b93450dcc18cede73b9d1ee7c6b0}{}\label{class_ui___widget_a3126b93450dcc18cede73b9d1ee7c6b0}
\index{Ui\+\_\+\+Widget@{Ui\+\_\+\+Widget}!label\+\_\+2@{label\+\_\+2}}
\index{label\+\_\+2@{label\+\_\+2}!Ui\+\_\+\+Widget@{Ui\+\_\+\+Widget}}
\subsubsection[{\texorpdfstring{label\+\_\+2}{label_2}}]{\setlength{\rightskip}{0pt plus 5cm}Q\+Label$\ast$ Ui\+\_\+\+Widget\+::label\+\_\+2}\hypertarget{class_ui___widget_a6f06b143349464b5b19ac0ffe2fc084d}{}\label{class_ui___widget_a6f06b143349464b5b19ac0ffe2fc084d}
\index{Ui\+\_\+\+Widget@{Ui\+\_\+\+Widget}!label\+\_\+3@{label\+\_\+3}}
\index{label\+\_\+3@{label\+\_\+3}!Ui\+\_\+\+Widget@{Ui\+\_\+\+Widget}}
\subsubsection[{\texorpdfstring{label\+\_\+3}{label_3}}]{\setlength{\rightskip}{0pt plus 5cm}Q\+Label$\ast$ Ui\+\_\+\+Widget\+::label\+\_\+3}\hypertarget{class_ui___widget_adfcab5569ac08da197e14dba01390755}{}\label{class_ui___widget_adfcab5569ac08da197e14dba01390755}
\index{Ui\+\_\+\+Widget@{Ui\+\_\+\+Widget}!label\+\_\+4@{label\+\_\+4}}
\index{label\+\_\+4@{label\+\_\+4}!Ui\+\_\+\+Widget@{Ui\+\_\+\+Widget}}
\subsubsection[{\texorpdfstring{label\+\_\+4}{label_4}}]{\setlength{\rightskip}{0pt plus 5cm}Q\+Label$\ast$ Ui\+\_\+\+Widget\+::label\+\_\+4}\hypertarget{class_ui___widget_a7d22bf9c5cf51754b1c145db5ca0da79}{}\label{class_ui___widget_a7d22bf9c5cf51754b1c145db5ca0da79}
\index{Ui\+\_\+\+Widget@{Ui\+\_\+\+Widget}!push\+Button@{push\+Button}}
\index{push\+Button@{push\+Button}!Ui\+\_\+\+Widget@{Ui\+\_\+\+Widget}}
\subsubsection[{\texorpdfstring{push\+Button}{pushButton}}]{\setlength{\rightskip}{0pt plus 5cm}Q\+Push\+Button$\ast$ Ui\+\_\+\+Widget\+::push\+Button}\hypertarget{class_ui___widget_a7dcf5da8902069415662905e93b0d5cb}{}\label{class_ui___widget_a7dcf5da8902069415662905e93b0d5cb}

\hypertarget{class_ui_1_1_widget}{}\section{Ui::Widget Class Reference}
\label{class_ui_1_1_widget}\index{Ui::Widget@{Ui::Widget}}


Inheritance diagram for Ui::Widget:\nopagebreak
\begin{figure}[H]
\begin{center}
\leavevmode
\includegraphics[width=142pt]{class_ui_1_1_widget__inherit__graph}
\end{center}
\end{figure}


Collaboration diagram for Ui::Widget:\nopagebreak
\begin{figure}[H]
\begin{center}
\leavevmode
\includegraphics[width=142pt]{class_ui_1_1_widget__coll__graph}
\end{center}
\end{figure}
\subsection*{Additional Inherited Members}

\hypertarget{class_widget}{}\section{Widget Class Reference}
\label{class_widget}\index{Widget@{Widget}}


Inheritance diagram for Widget\+:\nopagebreak
\begin{figure}[H]
\begin{center}
\leavevmode
\includegraphics[width=135pt]{class_widget__inherit__graph}
\end{center}
\end{figure}


Collaboration diagram for Widget\+:
\nopagebreak
\begin{figure}[H]
\begin{center}
\leavevmode
\includegraphics[width=216pt]{class_widget__coll__graph}
\end{center}
\end{figure}
\subsection*{Public Slots}
\begin{DoxyCompactItemize}
\item 
void \hyperlink{class_widget_adafcde4720aa7cb21f5d8966ae03c08c}{Print} ()
\end{DoxyCompactItemize}
\subsection*{Public Member Functions}
\begin{DoxyCompactItemize}
\item 
\hyperlink{class_widget_a29531c7f141e461322981b3b579d4590}{Widget} (Q\+Widget $\ast$parent=0)
\item 
\hyperlink{class_widget_aa24f66bcbaaec6d458b0980e8c8eae65}{$\sim$\+Widget} ()
\end{DoxyCompactItemize}
\subsection*{Private Attributes}
\begin{DoxyCompactItemize}
\item 
\hyperlink{class_ui_1_1_widget}{Ui\+::\+Widget} $\ast$ \hyperlink{class_widget_a19c48cc897c43aa2e995fce9f7fb2418}{ui}
\end{DoxyCompactItemize}


\subsection{Constructor \& Destructor Documentation}
\index{Widget@{Widget}!Widget@{Widget}}
\index{Widget@{Widget}!Widget@{Widget}}
\subsubsection[{\texorpdfstring{Widget(\+Q\+Widget $\ast$parent=0)}{Widget(QWidget *parent=0)}}]{\setlength{\rightskip}{0pt plus 5cm}Widget\+::\+Widget (
\begin{DoxyParamCaption}
\item[{Q\+Widget $\ast$}]{parent = {\ttfamily 0}}
\end{DoxyParamCaption}
)\hspace{0.3cm}{\ttfamily [explicit]}}\hypertarget{class_widget_a29531c7f141e461322981b3b579d4590}{}\label{class_widget_a29531c7f141e461322981b3b579d4590}


Here is the call graph for this function\+:\nopagebreak
\begin{figure}[H]
\begin{center}
\leavevmode
\includegraphics[width=350pt]{class_widget_a29531c7f141e461322981b3b579d4590_cgraph}
\end{center}
\end{figure}


\index{Widget@{Widget}!````~Widget@{$\sim$\+Widget}}
\index{````~Widget@{$\sim$\+Widget}!Widget@{Widget}}
\subsubsection[{\texorpdfstring{$\sim$\+Widget()}{~Widget()}}]{\setlength{\rightskip}{0pt plus 5cm}Widget\+::$\sim$\+Widget (
\begin{DoxyParamCaption}
{}
\end{DoxyParamCaption}
)}\hypertarget{class_widget_aa24f66bcbaaec6d458b0980e8c8eae65}{}\label{class_widget_aa24f66bcbaaec6d458b0980e8c8eae65}


\subsection{Member Function Documentation}
\index{Widget@{Widget}!Print@{Print}}
\index{Print@{Print}!Widget@{Widget}}
\subsubsection[{\texorpdfstring{Print}{Print}}]{\setlength{\rightskip}{0pt plus 5cm}void Widget\+::\+Print (
\begin{DoxyParamCaption}
{}
\end{DoxyParamCaption}
)\hspace{0.3cm}{\ttfamily [slot]}}\hypertarget{class_widget_adafcde4720aa7cb21f5d8966ae03c08c}{}\label{class_widget_adafcde4720aa7cb21f5d8966ae03c08c}


Here is the caller graph for this function\+:\nopagebreak
\begin{figure}[H]
\begin{center}
\leavevmode
\includegraphics[width=274pt]{class_widget_adafcde4720aa7cb21f5d8966ae03c08c_icgraph}
\end{center}
\end{figure}




\subsection{Member Data Documentation}
\index{Widget@{Widget}!ui@{ui}}
\index{ui@{ui}!Widget@{Widget}}
\subsubsection[{\texorpdfstring{ui}{ui}}]{\setlength{\rightskip}{0pt plus 5cm}{\bf Ui\+::\+Widget}$\ast$ Widget\+::ui\hspace{0.3cm}{\ttfamily [private]}}\hypertarget{class_widget_a19c48cc897c43aa2e995fce9f7fb2418}{}\label{class_widget_a19c48cc897c43aa2e995fce9f7fb2418}

%--- End generated contents ---

\clearpage


\part{Human Interface Design}

    \chapter{Overview of Human Interface}

    \chapter{Screen Images}

    \chapter{Screen Objects and Actions}

\clearpage


\part{Design Patterns}

\clearpage


\begin{thebibliography}{9}

\bibitem{example}
This is an example.

\end{thebibliography}


\end{document}
